\newif\ifshowsolutions
\showsolutionsfalse
\documentclass{article}
\usepackage{listings}
\usepackage{amsmath}
\usepackage{subfig}
\usepackage{amsthm}
\usepackage{amsmath}
\usepackage{amssymb}
\usepackage{graphicx}
\usepackage{mdwlist}
\usepackage{geometry}
\usepackage{titlesec}
\usepackage{palatino}
\usepackage{mathpazo}
\usepackage{fancyhdr}
\usepackage{paralist}
\usepackage{todonotes}
\usepackage{tikz}
\usepackage{float} % Place figures where you ACTUALLY want it
\usepackage{comment} % A hack to toggle sections
\usepackage{ifthen}
\usepackage{mdframed}
\usepackage{verbatim}
\usepackage{listings}
\usepackage{bbm}
\usepackage{upquote} % Prevents backticks replacing single-quotes in verbatim
\usepackage{listings}
\lstset{basicstyle=\ttfamily,
  showstringspaces=false,
  commentstyle=\color{red},
  keywordstyle=\color{blue}
}
\usepackage[strings]{underscore}
\usepackage[colorlinks=true]{hyperref}
\usetikzlibrary{positioning,shapes,backgrounds}

\geometry{margin=1in}
\geometry{headheight=2in}
\geometry{top=2in}

\setlength{\marginparwidth}{2.15cm}
\setlength{\parindent}{0em}
\setlength{\parskip}{0.6\baselineskip}

\rhead{}
\lhead{}

% Spacing settings.
\titlespacing\section{0pt}{12pt plus 2pt minus 2pt}{0pt plus 2pt minus 2pt}
\titlespacing\subsection{0pt}{12pt plus 4pt minus 2pt}{0pt plus 2pt minus 2pt}
\titlespacing\subsubsection{0pt}{12pt plus 4pt minus 2pt}{0pt plus 2pt minus 2pt}
\renewcommand{\baselinestretch}{1.15}

% Shortcuts for commonly used operators.
\newcommand{\E}{\mathbb{E}}
\newcommand{\Var}{\operatorname{Var}}
\newcommand{\Cov}{\operatorname{Cov}}
\newcommand{\Bias}{\operatorname{Bias}}
\DeclareMathOperator{\argmin}{arg\,min}
\DeclareMathOperator{\argmax}{arg\,max}

% Do not number subsections and below.
\setcounter{secnumdepth}{1}

% Custom format subsection.
\titleformat*{\subsection}{\large\bfseries}

% Set up the problem environment.
\newcounter{problem}[section]
\newenvironment{problem}[1][]
{\begingroup
  \setlength{\parskip}{0em}
  \refstepcounter{problem}\par\addvspace{1em}\textbf{Problem~\Alph{problem}\!
    \ifthenelse{\equal{#1}{}}{}{ [#1 points]}:}
  \endgroup}

% Set up the subproblem environment.
\newcounter{subproblem}[problem]
\newenvironment{subproblem}[1][]
{\begingroup
  \setlength{\parskip}{0em}
  \refstepcounter{subproblem}\par\medskip\textbf{\roman{subproblem}.\!
    \ifthenelse{\equal{#1}{}}{}{ [#1 points]:}}
  \endgroup}

% Set up the teachers and materials commands.
\newcommand\teachers[1]
{\begingroup
  \setlength{\parskip}{0em}
  \vspace{0.3em} \textit{\hspace*{2em} TAs responsible: #1} \par
  \endgroup}
\newcommand\materials[1]
{\begingroup
  \setlength{\parskip}{0em}
  \textit{\hspace*{2em} Relevant materials: #1} \par \vspace{1em}
  \endgroup}

% Set up the hint environment.
\newenvironment{hint}[1][]
{\begin{em}\textbf{Hint: }}
    {\end{em}}


% Set up the solution environment.
\ifshowsolutions
  \newenvironment{solution}[1][]
  {\par\medskip \begin{mdframed}\textbf{Solution~\Alph{problem}#1:} \begin{em}}
        {\end{em}\medskip\end{mdframed}\medskip}
  \newenvironment{subsolution}[1][]
  {\par\medskip \begin{mdframed}\textbf{Solution~\Alph{problem}#1.\roman{subproblem}:} \begin{em}}
        {\end{em}\medskip\end{mdframed}\medskip}
\else
  \excludecomment{solution}
  \excludecomment{subsolution}
\fi


\chead{
  {\vbox{
      \vspace{2mm}
      \large
      Computational Physics II \hfill
      UCSD PHYS 142/242 \hfill \\[1pt]
      Quiz 1\hfill
      Due: Friday, January 10, 2025, 8:00pm\\
    }
  }
}

\begin{document}
\pagestyle{fancy}


\section*{Policies}
Please work on the quiz individually.

\section*{Submission Instructions}
Please submit your quiz as a single .pdf file to Gradescope under ``Quiz 1".

\section{Classical action [20 Points]}
\materials{Week 1 lectures}
A free electron is moving in one dimension between position $x_A=0$ at time $t_A$ and position $x_B=0.15\times 10^{-8}\,\mathrm{cm}$ at time $t_B$.
The mass of the electron is $m=0.5\,\mathrm{MeV}/c^2 = 9.1\times10^{-28}\,\mathrm{g}$ where $c=3\times10^{10}\mathrm{cm/s}$ is the speed of light.
The electron is moving with (constant) velocity $v=0.15c$.

\begin{problem}[10]
Show that the classical action $S = \int_{t_A}^{t_B} L dt$ for a free particle ($L = \frac{1}{2}m\dot{x}^2$) is given by
\begin{equation}
  S = \frac{m}{2}\frac{(x_B-x_A)^2}{t_B-t_A} =  \frac{mv(x_B-x_A)}{2}.
\end{equation}
Evaluate the action $S$ in units of MeV\,s or cm$^2$\,g/s.
\end{problem}

\begin{solution}

\end{solution}

\begin{problem}[5]
Evaluate the action $S$ in units of $\hbar = h/(2\pi) = 6.5822\times10^{-22}\,\mathrm{MeV\,s} = 1.0546 \times 10^{-27}\,\mathrm{cm}^2\,\mathrm{g/s}$.
\end{problem}

\begin{solution}

\end{solution}


\begin{problem}[5]
Evaluate the action $S$ if the electron mass was 1\,g and the separation between $x_A$ and $x_B$ was 1\,cm.
Evaluate $S$ again in units of $\hbar$, like in (B).
What is the most striking difference between the two results in (B) and (C) in $\hbar$ units?
What does this mean for the classical limit of quantum mechanics, i.e. why do we not need to sum over all paths in classical mechanics?
\end{problem}

\begin{solution}

\end{solution}

\end{document}
