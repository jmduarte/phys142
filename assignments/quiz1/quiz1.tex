\newif\ifshowsolutions
\showsolutionsfalse
\input{../common/preamble}

\chead{
  {\vbox{
      \vspace{2mm}
      \large
      Computational Physics II \hfill
      UCSD PHYS 142/242 \hfill \\[1pt]
      Quiz 1\hfill
      Due: Friday, January 10, 2025, 8:00pm\\
    }
  }
}

\begin{document}
\pagestyle{fancy}


\section*{Policies}
Please work on the quiz individually.

\section*{Submission Instructions}
Please submit your quiz as a single .pdf file to Gradescope under ``Quiz 1".

\section{Classical action [20 Points]}
\materials{Week 1 lectures}
A free electron is moving in one dimension between position $x_A=0$ at time $t_A$ and position $x_B=0.15\times 10^{-8}\,\mathrm{cm}$ at time $t_B$.
The mass of the electron is $m=0.5\,\mathrm{MeV}/c^2 = 9.1\times10^{-28}\,\mathrm{g}$ where $c=3\times10^{10}\mathrm{cm/s}$ is the speed of light.
The electron is moving with (constant) velocity $v=0.15c$.

\begin{problem}[10]
Show that the classical action $S = \int_{t_A}^{t_B} L dt$ for a free particle ($L = \frac{1}{2}m\dot{x}^2$) is given by
\begin{equation}
  S = \frac{m}{2}\frac{(x_B-x_A)^2}{t_B-t_A} =  \frac{mv(x_B-x_A)}{2}.
\end{equation}
Evaluate the action $S$ in units of MeV\,s or cm$^2$\,g/s.
\end{problem}

\begin{solution}

\end{solution}

\begin{problem}[5]
Evaluate the action $S$ in units of $\hbar = h/(2\pi) = 6.5822\times10^{-22}\,\mathrm{MeV\,s} = 1.0546 \times 10^{-27}\,\mathrm{cm}^2\,\mathrm{g/s}$.
\end{problem}

\begin{solution}

\end{solution}


\begin{problem}[5]
Evaluate the action $S$ if the electron mass was 1\,g and the separation between $x_A$ and $x_B$ was 1\,cm.
Evaluate $S$ again in units of $\hbar$, like in (B).
What is the most striking difference between the two results in (B) and (C) in $\hbar$ units?
What does this mean for the classical limit of quantum mechanics, i.e. why do we not need to sum over all paths in classical mechanics?
\end{problem}

\begin{solution}

\end{solution}

\end{document}
